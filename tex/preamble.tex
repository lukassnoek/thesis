\usepackage{caption}
\usepackage[pass]{geometry} % load the package, but none of the default settings
\usepackage{pdflscape} % landscape pages and rotation
\usepackage{bookmark} % section links in pdf
\usepackage{enumitem} % formatting lists
\usepackage[dvipsnames]{xcolor} % prevent error in pdfpages
\usepackage{pdfpages} % for adding the book cover

% LS: add stuff for kableExtra
\usepackage{booktabs}
\usepackage{longtable}
\usepackage{array}
\usepackage{multirow}
\usepackage{wrapfig}
\usepackage{float}
\usepackage{colortbl}
\usepackage{pdflscape}
\usepackage{tabu}
\usepackage{threeparttable}
\usepackage{threeparttablex}
\usepackage[normalem]{ulem}
\usepackage{makecell}
\usepackage{xcolor}

\let\oldhref\href % save command before redefininig, so we can turn it off again
\renewcommand{\href}[2]{#2\footnote{\url{#1}}} % same as "link-as-notes: true"

\newcommand{\blandscape}{\begin{landscape}}
\newcommand{\elandscape}{\end{landscape}}
\AtBeginDocument{\let\maketitle\relax} % don't make automatic title page as first page

\newcommand{\CoverName}{cover} % to set page numbers of cover pages to "cover"

% change toc depth (to remove subsections in Dutch Summary from toc)
\newcommand{\changelocaltocdepth}[1]{%
  \addtocontents{toc}{\protect\setcounter{tocdepth}{#1}}%
  \setcounter{tocdepth}{#1}%
}

% From https://tex.stackexchange.com/questions/32547/how-to-measure-the-width-of-a-longtable-dynamically-and-use-this-width-in-footer
% papaja requires the \getlongtablewidth command when using the longtable=TRUE option with apa_table
\makeatletter
\newcommand\LastLTentrywidth{1em}
\newlength\longtablewidth
\setlength{\longtablewidth}{1in}
\newcommand{\getlongtablewidth}{\begingroup \ifcsname LT@\roman{LT@tables}\endcsname \global\longtablewidth=0pt \renewcommand{\LT@entry}[2]{\global\advance\longtablewidth by ##2\relax\gdef\LastLTentrywidth{##2}}\@nameuse{LT@\roman{LT@tables}} \fi \endgroup}
\makeatother

%% Typefaces
\usepackage{fontspec}
\setmainfont{Minion Pro}
\setsansfont[Ligatures=TeX]{Helvetica}
\setmathsfont(Digits,Greek,Latin)[Numbers={Proportional}]{Minion Pro}
\setmathrm{Minion Pro}

%% Page layout

% The length of the lowercase alphabet in 11 pt Minion Pro is 127.80293pt (116.7151 pt for 10pt).
% According to equation 2.1 of the memoir manual, the optimal width of the typeblock (66 characters) would then be 294.38358306 pt, or 103.9 mm
% According to table 2.2 of the memoir manual, the typeblock should be between 22 pica's (= 93.13 mm), which would be 59 characters wide, or 26 pica's a little bit more than 22 pica's (= 110.1 mm), which would be 70 characters wide. 

\setstocksize{240mm}{170mm} % adjusted B5, with no bleed on each side for now
\settrimmedsize{240mm}{170mm}{*} % adjusted B5 (standard thesis size)
\setpageml{\paperheight}{\paperwidth}{*} % center the adjusted B5 page to the middle left (so the right, bottom and top will be trimmed)
\settypeblocksize{*}{105mm}{1.618} % typeblock of 105 mm wide (little wider than optimal, to save paper). The golden ratio is a good rule to set the height, which amounts to about 105*1.618 = 170. The actual height of the text block will differ slightly, because it has to fit an integer number of lines.
\setlrmargins{*}{40mm}{*} % leaves 170-105 = 65 mm for margins. Set the foredge margin so its relation to the spine margin is about the golden ratio as well (65/1.618 is 40.2). A foreedge of twice the spine is also common, but I think this makes the foreedge a bit too big. The spine is then 65-40 = 25 mm.
\setulmargins{25mm}{*}{*} % the top margin is often 1/9 of the page height, or 1/9 * 240 mm = 27 mm. Often the top margin is also the same as the spine. Both of these rules almost converge here. 
% This automatically determines the bottom margin at 240 - 25 - 170 = 45. This is bigger than the top margin, which is good, as this is where you hold the book (often the bottom margin is even twice as big as the top).
%\setheadfoot{\onelineskip}{2\onelineskip} % defaults from memoir manual
%\setheaderspaces{*}{2\onelineskip}{*} % defaults from memoir manual
\setmarginnotes{5mm}{15mm}{\onelineskip} % too narrow (15mm, with 5 mm separation from text) for actual margin notes; but some chapter / pagestyles (e.g. companion) run off the page if this is not set.
\settypeoutlayoutunit{mm} % use mm for printing to the log
\checkandfixthelayout

%% Page and chapter styles
\chapterstyle{companion}

\def\defstyle{Ruled} % pagestyle set here will be used on all pages with regular text
\makeevenhead{Ruled}{\leftmark}{}{} % get rid of small caps in verso headers

% distinguish lower-level headings
\setsubsubsecheadstyle{\normalsize\bfseries\itshape} % bold italic
\setparaheadstyle{\normalsize\itshape} % just italic

%% Spacing
\midsloppy % middle ground between overfull lines ("fussy"), or lots of variation in interword spaces ("sloppy")

% try to avoid widows (last line of paragraph at the top of otherwise empty page)
\setlength{\topskip}{1.6\topskip}
\checkandfixthelayout
\sloppybottom

\renewcommand{\arraystretch}{1.5} % increase vertical spacing in tables

% when figures/tables are wider than text block, center them with respect to text block
\setfloatadjustment{figure}{\centerfloat}
\setfloatadjustment{table}{\centerfloat}

%\setlength{\cftpartnumwidth}{2.5em} % more space for part numbering in ToC
%\setlength\cftchapternumwidth{2.5em} % also adjust others to match
%\setlength\cftsectionindent{2.5em} % also adjust others to match

%% Abstracts
\abstractrunin % set "ABSTRACT" title in-line
\renewcommand{\abstractnamefont}{\normalfont\normalsize\scshape\bfseries} % set "ABSTRACT" in bold small scaps
\renewcommand{\abstracttextfont}{\normalfont\normalsize} % normalsize for text
\setlength{\absleftindent}{0pt} % don't increase left margin for abstracts
\setlength{\absrightindent}{0pt} % don't increase right margin for abstracts
\abslabeldelim{\quad} % some space after "ABSTRACT"
% see front_matter.tex for two more commands

%% Part titles 
%format similarly to companion chapter style
\renewcommand{\partnamefont}{\normalfont\huge\scshape}
\renewcommand{\partnumfont}{\normalfont\HUGE}

%\setlength\beforechapskip{-\baselineskip} % this would remove any space above the chapter titles

%% Captions
% Memoir:
%\captionnamefont{\small\sffamily}
%\captiontitlefont{\small\sffamily}
%\hangcaption
% caption package:
\captionsetup{labelfont={rm,sc},textfont={small,sf},labelsep=space} % small sans for table captions, with small caps for label

%% Bibliography
%\renewcommand*{\bibfont}{\footnotesize}  % lukas: commented out because error with none citationpackage

% Hyperlink entire author-year citation; not just year
%https://tex.stackexchange.com/questions/15951/hyperlink-name-with-biblatex-authoryear-biblatex-1-4b
%\DeclareFieldFormat{citehyperref}{%
%  \DeclareFieldAlias{bibhyperref}{noformat}% Avoid nested links
%  \bibhyperref{#1}}
%\DeclareFieldFormat{textcitehyperref}{%
%  \DeclareFieldAlias{bibhyperref}{noformat}% Avoid nested links
%  \bibhyperref{%
%    #1%
%    \ifbool{cbx:parens}
%      {\bibcloseparen\global\boolfalse{cbx:parens}}
%      {}}}
%\savebibmacro{cite}
%\savebibmacro{textcite}
%\renewbibmacro*{cite}{%
%  \printtext[citehyperref]{%
%    \restorebibmacro{cite}%
%    \usebibmacro{cite}}}
%\renewbibmacro*{textcite}{%
%  \ifboolexpr{
%    ( not test {\iffieldundef{prenote}} and
%      test {\ifnumequal{\value{citecount}}{1}} )
%    or
%    ( not test {\iffieldundef{postnote}} and
%      test {\ifnumequal{\value{citecount}}{\value{citetotal}}} )
%  }
%    {\DeclareFieldAlias{textcitehyperref}{noformat}}
%    {}%
%  \printtext[textcitehyperref]{%
%    \restorebibmacro{textcite}%
%    \usebibmacro{textcite}}}

% Handle prefixes in surnames (e.g. "van", "de") correctly
%https://tex.stackexchange.com/questions/440133/prefixes-in-author-names-in-references-and-bibliography
%\DeclareSortingNamekeyTemplate{
%  \keypart{
%    \namepart{family}
%  }
%  \keypart{
%    \namepart{prefix}
%  }
%  \keypart{
%    \namepart{given}
%  }
%  \keypart{
%    \namepart{suffix}
%  }
%}

%\renewbibmacro{begentry}{\midsentence}