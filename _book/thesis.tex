% This is the default LaTeX template (default-1.17.0.2.tex) from the RMarkdown package, from:
% https://github.com/rstudio/rmarkdown/blob/master/inst/rmd/latex/default-1.17.0.2.tex
%
% New additions to the template are marked with "LCR"

%\documentclass[12pt,american,]{book} %LCR
 % if not, force the oneside and a4paper options, which seem to be the only reasonable defaults
\documentclass[12pt,american,a4paper,oneside,]{book} %LCR

\usepackage{lmodern}
\usepackage{amssymb,amsmath}
\usepackage{ifxetex,ifluatex}
\usepackage{fixltx2e} % provides \textsubscript
\ifnum 0\ifxetex 1\fi\ifluatex 1\fi=0 % if pdftex
  \usepackage[T1]{fontenc}
  \usepackage[utf8]{inputenc}
\else % if luatex or xelatex
  \ifxetex
    \usepackage{mathspec}
  \else
    \usepackage{fontspec}
  \fi
  \defaultfontfeatures{Ligatures=TeX,Scale=MatchLowercase}
\fi
% use upquote if available, for straight quotes in verbatim environments
\IfFileExists{upquote.sty}{\usepackage{upquote}}{}
% use microtype if available
\IfFileExists{microtype.sty}{%
\usepackage{microtype}
\UseMicrotypeSet[protrusion]{basicmath} % disable protrusion for tt fonts
}{}

 %LCR
\usepackage{hyperref}
\hypersetup{unicode=true,
            pdftitle={Learning from the brain},
            pdfauthor={Lukas Snoek},
            pdfborder={0 0 0},
            breaklinks=true}
\urlstyle{same}  % don't use monospace font for urls
\ifnum 0\ifxetex 1\fi\ifluatex 1\fi=0 % if pdftex
  \usepackage[shorthands=off,main=american]{babel}
\else
  \usepackage{polyglossia}
  \setmainlanguage[variant=american]{english}
\fi
\usepackage{longtable,booktabs}
% Make links footnotes instead of hotlinks:
\renewcommand{\href}[2]{#2\footnote{\url{#1}}}
\IfFileExists{parskip.sty}{%
\usepackage{parskip}
}{% else
\setlength{\parindent}{0pt}
\setlength{\parskip}{6pt plus 2pt minus 1pt}
}
\setlength{\emergencystretch}{3em}  % prevent overfull lines
\providecommand{\tightlist}{%
  \setlength{\itemsep}{0pt}\setlength{\parskip}{0pt}}
\setcounter{secnumdepth}{5}
% Redefines (sub)paragraphs to behave more like sections
\ifx\paragraph\undefined\else
\let\oldparagraph\paragraph
\renewcommand{\paragraph}[1]{\oldparagraph{#1}\mbox{}}
\fi
\ifx\subparagraph\undefined\else
\let\oldsubparagraph\subparagraph
\renewcommand{\subparagraph}[1]{\oldsubparagraph{#1}\mbox{}}
\fi

%%% Use protect on footnotes to avoid problems with footnotes in titles
\let\rmarkdownfootnote\footnote%
\def\footnote{\protect\rmarkdownfootnote}

%%% This fixes a TexLive 2019 change that broke pandoc template. Will also be fixed in pandoc 2.8 %LCR
% https://github.com/jgm/pandoc/issues/5801
\renewcommand{\linethickness}{0.05em}


%%%%%%%%%%%%% BEGIN DOCUMENT %%%%%%%%%%%%%
\begin{document}

%% Page I: the half-title / "Franse pagina" %LCR
\frontmatter
\thispagestyle{empty}
\def\drop{.1\textheight}

\vspace*{\drop}
\begin{center}
\Huge \textsc{Learning from the brain}
\end{center}

%% Page II: Colophon %LCR
\clearpage
\thispagestyle{empty}
\vspace*{\fill}
\begingroup % to change formatting only temporarily
\small
\setlength{\parskip}{\baselineskip} % add space between paragraphs
\setlength\parindent{0pt} % no indents

This thesis was typeset using (R) Markdown, \LaTeX\ and the \verb+bookdown+ R-package
\\ ISBN: xxx-xx-xxxx-xxx-x\\ Printing: Acme Press, Inc.

An online version of this thesis is available at \url{https://lukas-snoek.com/thesis}, licensed under a CC BY.
\endgroup

%% Page III: `Title page' mandated by University of Amsterdam %LCR
\clearpage
\thispagestyle{empty}
\vspace*{\drop}
\begin{center}
\Huge\textbf{Learning from the brain}\par
\vspace{\baselineskip}
\Large\textit{Best practices for the use of neuroimaging data\\
in psychology research}\par
\vfill % this space will be whatever is left on the page
\large \textsc{Academisch Proefschrift}\par
\vspace{\baselineskip}
\linespread{1.3}{\normalsize ter verkrijging van de graad van doctor\\
aan de Universiteit van Amsterdam\\
op gezag van de Rector Magnificus\\
prof. dr. ir. K.I.J. Maex\\ % make sure this is the current rector magnificus
\mbox{ten overstaan van een door het College voor Promoties ingestelde commissie,}\\
in het openbaar te verdedigen in de Agnietenkapel\\
op maandag 21 oktober 2021, te 14 uur \\ }\par %
\vspace{\baselineskip}
{\large door}\par
\vspace{\baselineskip}
{\Large Lukas Snoek}\par
\vspace{\baselineskip}
{\large geboren te Hoevelaken}
\end{center}

%% Page IV: info on thesis committee %LCR
\clearpage
\thispagestyle{empty}
\noindent\textbf{Promotiecommissie:}\\
\\
\noindent\begin{tabular}{@{}lll}

Promotor:
&  dr. H.S. Scholte & Universiteit van Amsterdam\\

Copromotor:
&  dr. S. Oosterwijk & Universiteit van Amsterdam\\

\\
Overige leden:
&  prof. dr. R.E. Jack & University of Glasgow\\
&  prof. dr. R.W. Goebel & Maastricht University\\
&  prof. dr. B.U. Forstmann & Universiteit van Amsterdam\\
&  prof. dr. A.H. Fischer & Universiteit van Amsterdam\\
&  prof. dr. D. Borsboom & Universiteit van Amsterdam\\
&  prof. dr. A.G. Sanfey & Radboud University\\
\end{tabular}\\

\noindent Faculteit: Faculteit der Maatschappij- en Gedragswetenschappen

%%%%%%%%%%%%%%%%%%


{
\setcounter{tocdepth}{1}
\tableofcontents
}
\mainmatter
\hypertarget{introduction}{%
\chapter{Introduction}\label{introduction}}

\emph{The first chapter of the thesis, which introduces your PhD project. The filler-text below was created with the \href{http://www.elsewhere.org/journal/pomo}{postmodernism generator}.}

\begin{center}\rule{0.5\linewidth}{0.5pt}\end{center}

\hypertarget{learning-from-the-brain}{%
\section{Learning from the brain}\label{learning-from-the-brain}}

When reading this thesis' title, some might think that it contains a typo. Scientists want to learn \emph{about} the brain, right? Not \emph{from} the brain. Well, yes, neuroscientists do. But I am a psychologist at heart, interested in human behavior, cognition, and above all, emotion. I'm interested in the mind, not the brain. I don't care about axons, neurotransmitters, and the basal ganglia. Sure, I do believe, like any proper scientist, that everything we feel, perceive, and do is instantiated in the brain, but I do not necessarily think that \emph{just} studying the brain in isolation is going to teach us anything useful about the human psyche. Mind you, in this PhD you'll find several studies that analyze \emph{brain} data, but realize that my ultimate goal has always been to understand the mind.

\hypertarget{shared-states-using-mvpa-to-test-neural-overlap-between-self-focused-emotion-imagery-and-other-focused-emotion-understanding}{%
\chapter{Shared states: using MVPA to test neural overlap between self-focused emotion imagery and other-focused emotion understanding}\label{shared-states-using-mvpa-to-test-neural-overlap-between-self-focused-emotion-imagery-and-other-focused-emotion-understanding}}

\chaptermark{Shared states}

\vspace*{\fill}

\begin{center}\rule{0.5\linewidth}{0.5pt}\end{center}

\small

\noindent
\emph{This chapter has been published as}: Oosterwijk, S.*, Snoek, L.*, Rotteveel, M., Barrett, L. F., \& Scholte, H. S. (2017). Shared states: using MVPA to test neural overlap between self-focused emotion imagery and other-focused emotion understanding. \emph{Social cognitive and affective neuroscience, 12}(7), 1025-1035.

* Shared first authorship
\newpage
\normalsize

\textbf{Abstract}

The present study tested whether the neural patterns that support imagining ``performing an action'', ``feeling a bodily sensation'' or ``being in a situation'' are directly involved in understanding \emph{other people's} actions, bodily sensations and situations. Subjects imagined the content of short sentences describing emotional actions, interoceptive sensations and situations (self-focused task), and processed scenes and focused on \emph{how} the target person was expressing an emotion, \emph{what} this person was feeling, and \emph{why} this person was feeling an emotion (other-focused task). Using a linear support vector machine classifier on brain-wide multi-voxel patterns, we accurately decoded each individual class in the self-focused task. When generalizing the classifier from the self-focused task to the other-focused task, we also accurately decoded whether subjects focused on the emotional actions, interoceptive sensations and situations of \emph{others}. These results show that the neural patterns that underlie self-imagined experience are involved in understanding the experience of other people. This supports the theoretical assumption that the basic components of emotion experience and understanding share resources in the brain.

\hypertarget{Sharedstates-introduction}{%
\section{Introduction}\label{Sharedstates-introduction}}

To navigate the social world successfully it is crucial to understand other people. But how do people generate meaningful representations of other people's actions, sensations, thoughts and emotions? The dominant view assumes that representations of other people's experiences are supported by the same neural systems as those that are involved in generating experience in the self (e.g., Gallese et al., \protect\hyperlink{ref-gallese2004unifying}{2004}; see for an overview Singer, \protect\hyperlink{ref-singer2012past}{2012}). We tested this principle of self-other neural overlap directly, using multi-voxel pattern analysis (MVPA), across three different aspects of experience that are central to emotions: actions, sensations from the body and situational knowledge.

In recent years, evidence has accumulated that suggests a similarity between the neural patterns representing the self and others. For example, a great variety of studies have shown that observing actions and sensations in other people engages similar neural circuits as acting and feeling in the self (see for an overview Bastiaansen et al., \protect\hyperlink{ref-bastiaansen2009evidence}{2009}). Moreover, an extensive research program on pain has demonstrated an overlap between the experience of physical pain and the observation of pain in other people, utilizing both neuroimaging techniques (e.g., Lamm et al., \protect\hyperlink{ref-lamm2011meta}{2011}) and analgesic interventions (e.g., Rütgen et al., \protect\hyperlink{ref-rutgen2015placebo}{2015}; Mischkowski et al., \protect\hyperlink{ref-mischkowski2016painkiller}{2016}). This process of ``vicarious experience'' or ``simulation'' is viewed as an important component of empathy (Carr et al., \protect\hyperlink{ref-carr2003neural}{2003}; Decety, \protect\hyperlink{ref-decety2011dissecting}{2011}; Keysers \& Gazzola, \protect\hyperlink{ref-keysers2014dissociating}{2014}). In addition, it is argued that mentalizing (e.g.~understanding the mental states of other people) involves the same brain networks as those involved in self-generated thoughts (Uddin et al., \protect\hyperlink{ref-uddin2007self}{2007}; Waytz \& Mitchell, \protect\hyperlink{ref-waytz2011two}{2011}). Specifying this idea further, a constructionist view on emotion proposes that both emotion experience and interpersonal emotion understanding are produced by the same large-scale distributed brain networks that support the processing of sensorimotor, interoceptive and situationally relevant information (Barrett \& Satpute, \protect\hyperlink{ref-barrett2013large}{2013}; Oosterwijk \& Barrett, \protect\hyperlink{ref-oosterwijk2014embodiment}{2014}). An implication of these views is that the representation of self- and other-focused emotional actions, interoceptive sensations and situations overlap in the brain.

\hypertarget{how-to-control-for-confounds-in-decoding-analyses-of-neuroimaging-data}{%
\chapter{How to control for confounds in decoding analyses of neuroimaging data}\label{how-to-control-for-confounds-in-decoding-analyses-of-neuroimaging-data}}

\hypertarget{the-amsterdam-open-mri-collection-a-set-of-multimodal-mri-datasets-for-individual-difference-analyses}{%
\chapter{The Amsterdam Open MRI Collection, a set of multimodal MRI datasets for individual difference analyses}\label{the-amsterdam-open-mri-collection-a-set-of-multimodal-mri-datasets-for-individual-difference-analyses}}

\hypertarget{choosing-to-view-morbid-information-involves-reward-circuitry}{%
\chapter{Choosing to view morbid information involves reward circuitry}\label{choosing-to-view-morbid-information-involves-reward-circuitry}}

\hypertarget{using-predictive-modeling-to-quantify-the-importance-and-limitations-of-action-units-in-emotion-perception}{%
\chapter{Using predictive modeling to quantify the importance and limitations of action units in emotion perception}\label{using-predictive-modeling-to-quantify-the-importance-and-limitations-of-action-units-in-emotion-perception}}

\hypertarget{comparing-models-of-dynamic-facial-expression-perception}{%
\chapter{Comparing models of dynamic facial expression perception}\label{comparing-models-of-dynamic-facial-expression-perception}}

\backmatter

\hypertarget{bibliography}{%
\chapter*{Bibliography}\label{bibliography}}
\addcontentsline{toc}{chapter}{Bibliography}

\markboth{\MakeUppercase{Bibliography}}{} % have to explicitly state what to put in the heading (bug in bookdown?)
%format the references so they have a hanging indent. Remove these (and the \endgroup command) if you want regular indentation.
\begingroup
\hspace{\parindent}
\setlength{\parindent}{-0.25in}
\setlength{\leftskip}{0.25in}
\setlength{\parskip}{0pt}

\hypertarget{refs}{}
\leavevmode\hypertarget{ref-barrett2013large}{}%
Barrett, L. F., \& Satpute, A. B. (2013). Large-scale brain networks in affective and social neuroscience: Towards an integrative functional architecture of the brain. \emph{Current Opinion in Neurobiology}, \emph{23}(3), 361--372.

\leavevmode\hypertarget{ref-bastiaansen2009evidence}{}%
Bastiaansen, J. A., Thioux, M., \& Keysers, C. (2009). Evidence for mirror systems in emotions. \emph{Philosophical Transactions of the Royal Society B: Biological Sciences}, \emph{364}(1528), 2391--2404.

\leavevmode\hypertarget{ref-carr2003neural}{}%
Carr, L., Iacoboni, M., Dubeau, M.-C., Mazziotta, J. C., \& Lenzi, G. L. (2003). Neural mechanisms of empathy in humans: A relay from neural systems for imitation to limbic areas. \emph{Proceedings of the National Academy of Sciences}, \emph{100}(9), 5497--5502.

\leavevmode\hypertarget{ref-decety2011dissecting}{}%
Decety, J. (2011). Dissecting the neural mechanisms mediating empathy. \emph{Emotion Review}, \emph{3}(1), 92--108.

\leavevmode\hypertarget{ref-gallese2004unifying}{}%
Gallese, V., Keysers, C., \& Rizzolatti, G. (2004). A unifying view of the basis of social cognition. \emph{Trends in Cognitive Sciences}, \emph{8}(9), 396--403.

\leavevmode\hypertarget{ref-keysers2014dissociating}{}%
Keysers, C., \& Gazzola, V. (2014). Dissociating the ability and propensity for empathy. \emph{Trends in Cognitive Sciences}, \emph{18}(4), 163--166.

\leavevmode\hypertarget{ref-lamm2011meta}{}%
Lamm, C., Decety, J., \& Singer, T. (2011). Meta-analytic evidence for common and distinct neural networks associated with directly experienced pain and empathy for pain. \emph{Neuroimage}, \emph{54}(3), 2492--2502.

\leavevmode\hypertarget{ref-mischkowski2016painkiller}{}%
Mischkowski, D., Crocker, J., \& Way, B. M. (2016). From painkiller to empathy killer: Acetaminophen (paracetamol) reduces empathy for pain. \emph{Social Cognitive and Affective Neuroscience}, \emph{11}(9), 1345--1353.

\leavevmode\hypertarget{ref-oosterwijk2014embodiment}{}%
Oosterwijk, S., \& Barrett, L. F. (2014). Embodiment in the construction of emotion experience and emotion understanding. \emph{Routledge Handbook of Embodied Cognition. New York: Routledge}, 250--260.

\leavevmode\hypertarget{ref-rutgen2015placebo}{}%
Rütgen, M., Seidel, E.-M., Silani, G., Riečansky, I., Hummer, A., Windischberger, C., Petrovic, P., \& Lamm, C. (2015). Placebo analgesia and its opioidergic regulation suggest that empathy for pain is grounded in self pain. \emph{Proceedings of the National Academy of Sciences}, \emph{112}(41), E5638--E5646.

\leavevmode\hypertarget{ref-singer2012past}{}%
Singer, T. (2012). The past, present and future of social neuroscience: A european perspective. \emph{Neuroimage}, \emph{61}(2), 437--449.

\leavevmode\hypertarget{ref-uddin2007self}{}%
Uddin, L. Q., Iacoboni, M., Lange, C., \& Keenan, J. P. (2007). The self and social cognition: The role of cortical midline structures and mirror neurons. \emph{Trends in Cognitive Sciences}, \emph{11}(4), 153--157.

\leavevmode\hypertarget{ref-waytz2011two}{}%
Waytz, A., \& Mitchell, J. P. (2011). Two mechanisms for simulating other minds: Dissociations between mirroring and self-projection. \emph{Current Directions in Psychological Science}, \emph{20}(3), 197--200.

\endgroup

\backmatter

\end{document}
